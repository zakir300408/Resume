\documentclass[10.2pt,a4paper,sans]{moderncv}        
\moderncvstyle{banking}                            
\moderncvcolor{blue}                               
\usepackage[scale=0.87]{geometry}
\usepackage[T1]{fontenc}
\usepackage[utf8]{inputenc}  % Ensure proper UTF-8 encoding
\usepackage{helvet}  % Load Helvetica font
\renewcommand{\familydefault}{\sfdefault}  % Set default font to sans-serif (Helvetica)

% Personal Information
\name{Zakir}{Ullah}
\title{Machine Learning Engineer}                               
\phone[mobile]{+8613472738184}                   
\email{zakir@sagemotion.com}                               
\social[linkedin]{zakirullah300}
\social[github]{zakir300408}

\begin{document}
	
	\makecvtitle
	
	\section{Summary}
	\cvitem{}{
		As a curious person, Machine Learning and robotics were a natural fit for me as a PhD researcher. The thrill of discovering, implementing, and developing new algorithms and techniques to automate, predict, and improve technology is what motivates me to work every day. I have a background in Mechanical Engineering and Machine Learning algorithms, and most recently, wearable systems and sensors. I have developed useful libraries, written predictive models for wearable systems to predict ground reaction forces from wearable IMUs, filtered EMG signals to measure muscle movement reactions, estimated step width based on IMUs from patients with ataxia, used various filters and physics-based equations to determine the orientation of different body segments from Quat data, and applied reinforcement learning and autoencoders for soft robotic control. I have also used computer vision-based ML models like Unet, Yolo, and various other models for segmentation, detection, tracking, and real-time calculations. I try to learn and adapt quickly; for example, I was recently introduced to JIRA, which my team uses, through which I have solved many issues, completed mergers, and collaborated extensively with my team on GitHub. As a result of my so-called startup mentality, I realized that my wide skill set would be better utilized in projects that involve a bit of the unknown and the mystery of novelty and development. Besides that, I am a very flexible person; having spent six years as a researcher makes you flexible anyway, and adaptability is a prerequisite for it.}	
	
	\section{Skills}
	\cvitem{Programming Languages}{6+ years of Python and C++ experience.}
	\cvitem{Machine Learning \& AI}{
		\begin{itemize}
			\item Deep Learning (PyTorch)
			\item Computer Vision (ResNet, UNet, YOLO)
			\item Time Series (LSTM, RandomForest)
			\item Reinforcement Learning (TD3, PPO)
			\item SVMs, K-NNs, clustering, autoencoders, boosting, bagging
			\item scikit-learn
		\end{itemize}
	}
	\cvitem{Tools \& Technologies}{
		\begin{itemize}
			\item Git, DVC
			\item Arduino, Raspberry Pi, STM32 programming
			\item CI/CD pipelines using GitHub
		\end{itemize}
	}
	\cvitem{Project Management}{Practical experience using Jira for project tracking and testing.}
	\cvitem{Design \& CAD}{
		\begin{itemize}
			\item Autodesk Inventor
			\item SolidWorks
			\item CAM for mechanical design and 3D modeling
		\end{itemize}
	}
	
	\section{Professional Experience}
	\cventry{Feb 2023--Present}{Machine Learning Engineer}{\textbf{SageMotion}}{}{}{
		\begin{itemize}%
			\item Developed and deployed a Machine Learning model to predict Ground Reaction Force (GRF) using wearable IMUs in real-time, now part of \textbf{SageMotion} System. 
			\item Designed an SDK in C++ for an EEG device's (\textbf{Niantong}) onboard controller and Python PyPi package; this SDK enabled \textbf{Niantong} customers to use the device with other hardware and streaming through Lab Streaming Layer (LSL).
			\item Worked on a real-time app that can predict step width from IMU data in ataxia patients and healthy subjects.
			\item Engineered a PCB test fixture for Bluetooth module re-flashing, cutting down programming time from 1 hour to 2 minutes.
			\item Developed an application for network communication between the Vicon system and \textbf{SageMotion} hub.
			\item Led the integration of EMG sensors with the \textbf{SageMotion} System for real-time data feedback-based training; used a low pass filter for data cleaning.
		\end{itemize}
	}
	\cvitem{}{Projects backed by GitHub repositories for verification and insights: \url{https://github.com/zakir300408}}
	
	\section{Education}
	\cventry{2022--2025}{PhD in Machine Learning and Wearable Systems}{Shanghai Jiao Tong University (SJTU)}{}{}{
		\begin{itemize}%
			\item Using reinforcement learning and other ML techniques to obtain magnetic soft robot control, focusing on TD3, PPO algorithms, and ensemble MLPs. [Paper Under Review in \textit{IEEE Transactions on Robotics}]
			\item Using ML to predict stride width and stride time variability in Ataxia patients, contributing to significant advancements in medical robotics. [Paper Under Review]
			\item Used Bidirectional autoencoders to generate synthetic data for electromagnetic signal data, resulting in joystick control [Paper Under Review]
		\end{itemize}
	}
	
	\cventry{2018--2022}{Master of Science in Mechanical Engineering}{Beihang University (BUAA)}{}{}{
		\begin{itemize}%
			\item Completed polyp segmentation using a Unet-based model for capsule robots; used GAN to create synthetic images to increase the minority classes, hence obtaining a 15\% improvement in segmentation performance.
			\item Designed a biopsy capsule robot for minimally invasive medical procedures within the gastrointestinal tract. \href{https://ieeexplore.ieee.org/abstract/document/8860958}{[Link to paper]}
		\end{itemize}
	}
	
	\cventry{2014--2018}{Bachelor of Science in Mechanical Engineering (With Distinction)}{Beihang University (BUAA)}{}{}{
		\begin{itemize}%
			\item Designed, manufactured, programmed, and controlled a novel non-motor-based electromagnetic swimming robot with a semi-flexible fin with Arduino-based control and Bluetooth connectivity; built an Android app for its control. \href{https://ieeexplore.ieee.org/document/9612684}{[Link to paper]}
			\item Various semester projects including a BB8 robot, an obstacle-avoiding robot, and an electric hybrid trike design.
		\end{itemize}
	}
	
	\section{Awards and Scholarships}
	\cventry{2022--2025}{Full PhD Scholarship}{Shanghai Jiao Tong University (SJTU)}{}{}{
		Awarded a full CSC scholarship at SJTU (45th QS ranking).
	}
	\cventry{2018--2022}{CSC Scholarship for Master's Studies}{Beihang University (BUAA)}{}{}{
		Awarded a full CSC scholarship for Master's due to receiving the Best Thesis Award during my bachelor's.
	}
	\cventry{2018}{Best Thesis Award for Bachelor's Thesis}{Beihang University (BUAA)}{}{}{
		Awarded for a non-motor-based electromagnetic swimming robot project that was the focus of my bachelor's thesis.
	}
	\cventry{2014--2018}{Full Scholarship for Bachelor's Studies}{Beihang University (BUAA)}{}{}{
		Received for excellent academic performance in high school, covering full tuition subject to maintaining a top 5 class ranking during bachelor's (Condition fulfilled).
	}
	\cventry{2015--2017}{Consecutive Excellent Study Awards}{Beihang University (BUAA)}{}{}{
		Awarded annually for top academic performance in the School of Mechanical Engineering.
	}
	
\end{document}
