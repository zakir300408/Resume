\documentclass[10pt,a4paper]{article}
\usepackage[utf8]{inputenc}
\usepackage[T1]{fontenc}
\usepackage[margin=1in]{geometry}
\usepackage{enumitem}
\usepackage{hyperref}
\usepackage{titlesec}
\usepackage{parskip}

% Set title format for sections for better visibility
\titleformat{\section}{\large\bfseries}{}{0em}{}[\titlerule]

% No indent
\setlength{\parindent}{0pt}

% Simple ATS-friendly style, better structure
\begin{document}
	
	% Header
	\begin{center}
		{\LARGE \textbf{Zakir Ullah}} \\
		\vspace{0.2cm}
		Machine Learning Engineer \\
		\vspace{0.2cm}
		\begin{tabular}{rl}
			Phone: & +8613472738184 \\
			Email: & \href{mailto:zakir@sagemotion.com}{zakir@sagemotion.com} \\
			LinkedIn: & \href{https://www.linkedin.com/in/zakirullah300}{linkedin.com/in/zakirullah300} \\
			GitHub: & \href{https://github.com/zakir300408}{github.com/zakir300408}
		\end{tabular}
	\end{center}
	
	\vspace{0.5cm}
	
	% Professional Experience Section
	\section{Professional Experience}
	
	\textbf{Machine Learning Engineer, SageMotion} \hfill \textit{Feb 2023 -- Oct 2024 (Hybrid, United States)} \\
	\begin{itemize}[left=0pt, nosep]
		\item Developed machine learning algorithms for real-time biomechanical data collection and prediction.
		\item Wrote C++ software for EEG and EMG devices, integrating with the Vicon system.
		\item Combined multiple sensor streams (IMU, EMG) for real-time feedback in biomechanics.
		\item Built a real-time EMG sensor app showcased at the ASB conference:
		
		 \href{https://github.com/SageMotionApps/00_xx_EMG_Game_ASB_Sagemotion}{github.com/SageMotionApps/00\_xx\_EMG\_Game\_ASB\_Sagemotion}.
		\item Developed a Linux-based real-time system to predict ground reaction forces using LSTM and sensor fusion:
		
		 \href{https://github.com/zakir300408/Ground_Reaction_Forces_Sagemotion}{github.com/zakir300408/Ground\_Reaction\_Forces\_Sagemotion}.
		\item Engineered a PCB fixture that reduced Bluetooth flashing time by 90\%.
		\item Increased Sage system's range from 1 meter to 9 meters for a client.
	\end{itemize}
	
	\textbf{Software Developer, Niantong Intelligence Company} \hfill \textit{July 2023 -- Oct 2023 (On-site, Shanghai, China)} \\
	\begin{itemize}[left=0pt, nosep]
		\item Developed a C++ SDK for EEG device communication via Bluetooth and Wifi, wrapped in Python:
		
		 \href{https://github.com/zakir300408/niantong_outsource_sdk}{github.com/zakir300408/niantong\_outsource\_sdk}.
		\item Published the SDK as a PyPI package:
		
		 \href{https://pypi.org/project/niantongEEG/}{pypi.org/project/niantongEEG}.
		\item Built a GUI for receiving and transmitting EEG data over the network using PyLSL:
		
		 \href{https://github.com/zakir300408/niantong_GUI}{github.com/zakir300408/niantong\_GUI}.
	\end{itemize}
	
	
	% Project Experience Section
	\section{Project Experience}
	\textbf{Real-Time Control of Magnetic Soft Robot using Reinforcement Learning, SJTU} \hfill \textit{2023 -- 2024} \\
	\begin{itemize}[left=0pt, nosep]
		\item Developed a control method using TD3 and PPO algorithms, achieving 97\% accuracy.
	\end{itemize}
	
	\textbf{EMG-based Machine Learning Model for Predicting Gait Parameters, SageMotion} \hfill \textit{2023} \\
	\begin{itemize}[left=0pt, nosep]
		\item Created an EMG-based model to predict gait events using leg-worn EMG sensors.
	\end{itemize}
	
	\textbf{Ground Reaction Force Prediction Using LSTM, SageMotion} \hfill \textit{2023} \\
	\begin{itemize}[left=0pt, nosep]
		\item Wrote an ML model using LSTM to predict ground reaction force (GRF) from IMU data with RMSE of 0.04.
	\end{itemize}
	
	% Education Section
	\section{Education}
	\textbf{Researcher in Machine Learning and Wearable Systems, Shanghai Jiao Tong University} \hfill \textit{2022 -- 2024} \\
	\begin{itemize}[left=0pt, nosep]
		\item Focused on reinforcement learning and magnetic soft robot control using TD3, PPO, and convolutional networks.
	\end{itemize}
	
	\textbf{Master of Science in Mechanical Engineering, Beihang University} \hfill \textit{2018 -- 2022} \\
	\begin{itemize}[left=0pt, nosep]
		\item Completed polyp segmentation using a UNet-based model for capsule robots, leveraging GAN to improve performance by 15\%.
	\end{itemize}
	
	\textbf{Bachelor of Science in Mechanical Engineering, Beihang University (With Distinction)} \hfill \textit{2014 -- 2018} \\
	\begin{itemize}[left=0pt, nosep]
		\item Designed, manufactured, and programmed a non-motor-based electromagnetic swimming robot, earning the Best Thesis Award.
	\end{itemize}
	
	% Skills Section
	\section{Skills}
	\begin{itemize}[left=0pt, nosep]
		\item \textbf{Programming Languages}: 6+ years of Python and C++ experience.
		\item \textbf{Machine Learning \& AI}:
		\begin{itemize}[left=0pt, nosep]
			\item Deep Learning (PyTorch), Computer Vision (ResNet, UNet, YOLO, CNN), Time Series (LSTM, RandomForest, RNN), Reinforcement Learning (TD3, PPO)
			\item SVMs, K-NNs, clustering, autoencoders, boosting, bagging, scikit-learn
		\end{itemize}
		\item \textbf{Tools \& Technologies}: Git, DVC, Jira, Arduino, Raspberry Pi, STM32 programming, CI/CD pipelines using GitHub
		\item \textbf{Project Management}: Practical experience using Jira for project tracking and testing.
		\item \textbf{Design \& CAD}: Autodesk Inventor, SolidWorks, CAM for mechanical design, 3D modeling, 4D printing, Direct ink writing for soft materials.
	\end{itemize}
	
	% Awards Section
	\section{Awards and Scholarships}
	\begin{itemize}[left=0pt, nosep]
		\item Full Research Scholarship, Shanghai Jiao Tong University (2022--2024)
		\item CSC Scholarship for Master's Studies, Beihang University (2018--2022)
		\item Best Thesis Award, Beihang University (2018)
	\end{itemize}
	
\end{document}
